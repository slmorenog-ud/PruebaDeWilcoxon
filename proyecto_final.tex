\documentclass[12pt, letterpaper]{article}

% --- PAQUETES NECESARIOS ---

% Codificación y fuentes
\usepackage[utf8]{inputenc}
\usepackage[spanish]{babel}
\usepackage{amsmath}
\usepackage{graphicx}
\usepackage{geometry}
\usepackage{hyperref}

% Configuración de márgenes
\geometry{
    letterpaper,
    top=2.5cm,
    bottom=2.5cm,
    left=3cm,
    right=3cm
}

% Paquete para bibliografía estilo APA
\usepackage{apacite}

% Paquete para insertar código fuente
\usepackage{listings}
\usepackage{xcolor}

% Configuración de 'listings' para el código de Python
\definecolor{codegreen}{rgb}{0,0.6,0}
\definecolor{codegray}{rgb}{0.5,0.5,0.5}
\definecolor{codepurple}{rgb}{0.58,0,0.82}
\definecolor{backcolour}{rgb}{0.95,0.95,0.92}

\lstdefinestyle{mystyle}{
    backgroundcolor=\color{backcolour},
    commentstyle=\color{codegreen},
    keywordstyle=\color{magenta},
    numberstyle=\tiny\color{codegray},
    stringstyle=\color{codepurple},
    basicstyle=\footnotesize\ttfamily,
    breakatwhitespace=false,
    breaklines=true,
    captionpos=b,
    keepspaces=true,
    numbers=left,
    numbersep=5pt,
    showspaces=false,
    showstringspaces=false,
    showtabs=false,
    tabsize=2
}
\lstset{style=mystyle}


% --- DATOS DEL DOCUMENTO ---
\title{
    \textbf{Un Título Ingenioso y Llamativo Sobre la Prueba de Wilcoxon} \\
    \large Proyecto Final
}
\author{
    ÁLVAREZ ORTIZ ARLEY SANTIAGO - 20241020008 \\
    MARTÍNEZ PARDO SILVANA - 20241020010 \\
    MORENO GRANADO SERGIO LEONARDO - 20242020091 \\
    RODRÍGUEZ CAMACHO JUAN ESTEBAN - 20241020029
}
\date{
    \textbf{Asignatura:} Probabilidad y Estadística \\
    \textbf{Docente:} Diego Alberto Chitiva Huertas \\
    \vspace{1cm}
    \textbf{Universidad Distrital Francisco José de Caldas} \\
    Facultad de Ingeniería \\
    Ingeniería de Sistemas \\
    \vspace{1cm}
    Miércoles, 10 de diciembre de 2025
}


% --- INICIO DEL DOCUMENTO ---
\begin{document}

\maketitle
\thispagestyle{empty}
\newpage
\tableofcontents
\newpage

% --- INTRODUCCIÓN ---
\section{Introducción}
\label{sec:introduccion}

En este apartado, se debe delimitar el tema central del proyecto: la prueba de rangos con signo de Wilcoxon para observaciones pareadas.

Se recomienda explicar brevemente qué es una prueba no paramétrica, en qué contextos se utiliza y por qué es relevante. Además, se deben establecer los objetivos del documento, como:
\begin{itemize}
    \item Presentar la fundamentación teórica de la prueba de Wilcoxon.
    \item Implementar un algoritmo en Python para automatizar el cálculo de la prueba.
    \item Demostrar su aplicación a través de ejemplos prácticos.
    \item Reflexionar sobre la importancia de esta herramienta estadística.
\end{itemize}
Aquí deben enganchar al lector, mostrando la relevancia del tema en el campo de la ingeniería o la ciencia de datos.


% --- DESARROLLO ---
\section{Desarrollo}
\label{sec:desarrollo}

Esta es la sección principal del documento. Se debe presentar de manera clara y ordenada la teoría y los ejemplos.

\subsection{Fundamentación Teórica}
Aquí se debe explicar en detalle la teoría detrás de la prueba de Wilcoxon, basándose en el libro de Wackerly, Mendenhall y Scheaffer (2008). Es importante incluir:
\begin{itemize}
    \item Las hipótesis nula ($H_0$) y alternativa ($H_a$).
    \item El procedimiento para calcular las diferencias, los rangos y el estadístico de prueba $T$.
    \item La distinción entre el manejo de muestras pequeñas (usando tablas de valores críticos) y muestras grandes (aproximación a la normal).
\end{itemize}
Recuerden citar correctamente la fuente. Por ejemplo: "Según Wackerly et al. (2008), el estadístico de prueba se calcula como...".

\subsection{Ejemplos y Problemas Resueltos}
En esta parte, deben desarrollar ejemplos paso a paso. Pueden tomar un problema del libro o plantear uno propio.

\subsubsection{Ejemplo con Muestra Pequeña}
Describir el problema, mostrar los datos, calcular las diferencias, asignar los rangos, obtener el valor de $T$ y tomar una decisión estadística comparando con el valor crítico de la tabla.

\subsubsection{Ejemplo con Muestra Grande}
Presentar un caso con un número de observaciones mayor (usualmente $n > 20$). Realizar el procedimiento y utilizar la aproximación a la distribución normal para tomar la decisión.


% --- ALGORITMO ---
\section{Algoritmo}
\label{sec:algoritmo}

En esta sección se presenta la implementación del algoritmo en Python. Se debe incluir el código fuente y una breve explicación de cómo funciona.

\subsection{Código Fuente en Python}
A continuación se muestra un esqueleto del código que pueden implementar.

\begin{lstlisting}[language=Python, caption=Algoritmo de Wilcoxon en Python]
import numpy as np
from scipy.stats import norm, wilcoxon

def prueba_wilcoxon_pareada(muestra1, muestra2, alpha=0.05):
    """
    Implementa la prueba de rangos con signo de Wilcoxon para
    observaciones pareadas.

    Argumentos:
    muestra1 -- Primera muestra de datos (lista o array de numpy).
    muestra2 -- Segunda muestra de datos (lista o array de numpy).
    alpha -- Nivel de significancia.

    Retorna:
    Un diccionario con el estadistico T, el p-valor y la decision.
    """

    # Validar que las muestras tengan el mismo tamano
    if len(muestra1) != len(muestra2):
        raise ValueError("Las muestras deben tener el mismo tamano.")

    # Calcular las diferencias y eliminar las diferencias nulas
    diferencias = np.array(muestra1) - np.array(muestra2)
    diferencias = diferencias[diferencias != 0]

    n = len(diferencias)

    # Usaremos la funcion de scipy para facilitar el calculo
    # pero aqui podrian implementar el calculo manual de rangos
    # si lo desean.

    # La funcion de scipy devuelve T y el p-valor
    T, p_valor = wilcoxon(diferencias)

    # Tomar la decision
    decision = "Rechazar H0" if p_valor < alpha else "No rechazar H0"

    return {
        "n_efectivo": n,
        "estadistico_T": T,
        "p_valor": p_valor,
        "decision": decision
    }

# --- Ejemplo de uso ---
# Datos de un problema hipotetico
datos_antes = [20, 25, 22, 28, 30, 24, 26, 27, 29, 23]
datos_despues = [18, 24, 20, 25, 28, 22, 23, 26, 27, 21]

resultado = prueba_wilcoxon_pareada(datos_antes, datos_despues)
print(resultado)
\end{lstlisting}

\subsection{Explicación del Algoritmo}
Aquí deben explicar qué hace cada parte del código: cómo se calculan las diferencias, cómo se manejan los ceros, y cómo se obtiene el resultado final. Pueden explicar la lógica detrás del cálculo de rangos si deciden implementarlo manualmente.


% --- CONCLUSIÓN ---
\section{Conclusión}
\label{sec:conclusion}

En la conclusión, deben resumir los puntos clave del documento. Hagan una síntesis de la teoría presentada y de los resultados obtenidos.

Finalmente, ofrezcan una reflexión sobre la importancia de la prueba de Wilcoxon. Por ejemplo, pueden discutir sobre su utilidad en situaciones donde los datos no siguen una distribución normal, que es un escenario común en problemas del mundo real. También pueden mencionar posibles extensiones o mejoras al trabajo realizado.


% --- BIBLIOGRAFÍA ---
\section{Bibliografía}
\label{sec:bibliografia}

Aquí se listan todas las fuentes citadas en el documento. Gracias al paquete `apacite`, las citas se formatearán automáticamente en estilo APA.

% Este comando lee el archivo .bib y genera la bibliografía
\bibliographystyle{apacite}
\bibliography{bibliografia}


\end{document}
