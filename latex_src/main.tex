%%----------------------------------------------------------------------------------
% Configuración del documento: A4, 11pt, impresión a una cara, estilo libro.
%%----------------------------------------------------------------------------------
\documentclass[a4paper,11pt,oneside]{book}

% --- PAQUETES BÁSICOS ---
\usepackage{fontspec}
\usepackage{polyglossia}
\setmainlanguage{spanish}
\setmonofont{JetBrains Mono}[Scale=MatchLowercase]

% --- PAQUETES DE FORMATO Y DISEÑO ---
\usepackage{geometry}
\geometry{
    a4paper,
    top=2.5cm,
    bottom=2.5cm,
    left=3cm,
    right=3cm
}

% --- PAQUETES MATEMÁTICOS Y CIENTÍFICOS ---
\usepackage{amsmath}
\usepackage{graphicx}

% --- NUMERACIÓN DE SECCIONES (sin número de capítulo) ---
\renewcommand{\thesection}{\arabic{section}}
\renewcommand{\thesubsection}{\thesection.\arabic{subsection}}

% --- PAQUETES PARA CÓDIGO Y COLORES ---
\usepackage{listings}
\usepackage{xcolor}

% Configuración de colores para listings
\definecolor{codebg}{rgb}{0.95,0.95,0.95}
\definecolor{codeframe}{rgb}{0.82,0.82,0.82}
\definecolor{keyword}{rgb}{0.0,0.2,0.65}
\definecolor{comment}{rgb}{0.25,0.5,0.35}
\definecolor{string}{rgb}{0.58,0.0,0.05}
\definecolor{linenumber}{rgb}{0.45,0.45,0.45}

\lstdefinestyle{mystyle}{
  backgroundcolor=\color{codebg},
  frame=single,
  rulecolor=\color{codeframe},
  framesep=6pt,
  framerule=0.6pt,
  basicstyle=\ttfamily\small,
  keywordstyle=\color{keyword}\bfseries,
  commentstyle=\itshape\color{comment},
  stringstyle=\color{string},
  numbers=left,
  numberstyle=\tiny\color{linenumber},
  stepnumber=1,
  numbersep=8pt,
  showstringspaces=false,
  breaklines=true,
  postbreak=\mbox{\textcolor{codeframe}{$\hookrightarrow$}\space},
  tabsize=2,
  captionpos=b,
  xleftmargin=6pt,
  xrightmargin=0pt
}

\lstset{style=mystyle, language=Python}

% --- PAQUETES AUXILIARES ---
\usepackage{caption}
\usepackage[style=apa, backend=biber]{biblatex}
\addbibresource{references.bib}

% --- HIPERVÍNCULOS (cargar al final) ---
\usepackage{hyperref}

%%%%%%%%%%%%%%%%%%%%%%%%%%%%%%%%%%%%%%%%%%%%%%%%%%%%%%%%%%%%%%%%%%%%%%%%%%%%%%%%%%%%
\begin{document}

\frontmatter

%%%%%%%%%%%%%%%%%%%%%%%%%%%%%%%%%%%%%%%%%%%%%%%%%%%%%%%%%%%%%%%%%%%%%%%%%%%%%%%%
% --- PORTADA ---
\begin{titlepage}
    \begin{center}
        \includegraphics[width=6cm]{figures/UDLogo.png}\\[0.5cm]
        {\LARGE Universidad Distrital Francisco José de Caldas\\[0.3cm]
        Facultad de Ingeniería\\[0.2cm]
        Ingeniería de Sistemas\\[2.5cm]
        }

        \linespread{1.0}\huge \textbf{\textit{
            Prueba de Rangos con Signo de Wilcoxon
            para Observaciones Pareadas
        }}
        \linespread{1}\\[2.5cm]

        {\Large
            Álvarez Ortiz Arley Santiago -- 20241020008 \\
            Martínez Pardo Silvana -- 20241020010 \\
            Moreno Granado Sergio Leonardo -- 20242020091 \\
            Rodríguez Camacho Juan Esteban -- 20241020029 \\[1cm]
        }

        {\Large
            \emph{Docente:} Diego Alberto Chitiva Huertas}\\[1.5cm]

        \large Proyecto Final de Probabilidad y Estadística\\
        Semestre 2025-3
        \\[0.3cm]
        \vfill

        Diciembre de 2025, Bogotá D.C.
    \end{center}
\end{titlepage}

\mainmatter

\tableofcontents
\newpage

% --- INTRODUCCIÓN ---
\section{Introducción}
\label{sec:introduccion}
En este apartado se debe delimitar el tema... (Texto de ejemplo)

% --- DESARROLLO ---
\section{Desarrollo}
\label{sec:desarrollo}
Esta es la sección principal...
\subsection{Fundamentación Teórica}
Aquí se debe explicar la teoría...
Citamos nuestra fuente de ejemplo \parencite{wackerly2008}.

% --- ALGORITMO ---
\section{Algoritmo}
\label{sec:algoritmo}
En esta sección se presenta la implementación...
\begin{lstlisting}[caption=Algoritmo de Wilcoxon en Python]
import numpy as np
from scipy.stats import wilcoxon

def prueba_wilcoxon_pareada(muestra1, muestra2, alpha=0.05):
    # Calcular las diferencias
    diferencias = np.array(muestra1) - np.array(muestra2)
    
    # Realizar la prueba de Wilcoxon
    estadistico, p_valor = wilcoxon(diferencias)
    
    # Tomar decisión
    if p_valor < alpha:
        decision = "Rechazar H0: Existen diferencias significativas"
    else:
        decision = "No rechazar H0: No hay diferencias significativas"
    
    return estadistico, p_valor, decision

# Ejemplo de uso
muestra_A = [23, 21, 25, 28, 22, 24, 27]
muestra_B = [20, 22, 24, 25, 21, 23, 26]

resultado = prueba_wilcoxon_pareada(muestra_A, muestra_B)
print(f"Estadístico: {resultado[0]:.3f}")
print(f"P-valor: {resultado[1]:.4f}")
print(f"Decisión: {resultado[2]}")
\end{lstlisting}

% --- CONCLUSIÓN ---
\section{Conclusión}
\label{sec:conclusion}
En la conclusión, deben resumir...

% --- BIBLIOGRAFÍA ---
\printbibliography[title={Bibliografía}]

\end{document}