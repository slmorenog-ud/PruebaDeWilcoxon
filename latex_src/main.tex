%%----------------------------------------------------------------------------------
% Configuracion del documento: A4, 11pt, impresion a una cara, estilo libro.
%%----------------------------------------------------------------------------------
\documentclass[a4paper,11pt,oneside]{book}

% --- PAQUETES ESENCIALES ---

% Codificacion y fuentes para espanol
\usepackage[utf8]{inputenc}
\usepackage[spanish]{babel}

% Paquetes para matematicas, graficos y geometria
\usepackage{amsmath}
\usepackage{graphicx} % Necesario para incluir imagenes (logo)
\usepackage{geometry}
\usepackage{hyperref}
\usepackage{caption}  % Para personalizar los pies de foto

% Configuracion de margenes
\geometry{
    a4paper,
    top=2.5cm,
    bottom=2.5cm,
    left=3cm,
    right=3cm
}

% --- GESTION DE BIBLIOGRAFIA CON BIBLATEX (ESTILO APA) ---
\usepackage[style=apa, backend=biber]{biblatex}
\addbibresource{references.bib} % Apunta al archivo .bib

% --- PAQUETE PARA INSERTAR CODIGO FUENTE ---
\usepackage{listings}
\usepackage{xcolor}

% Configuracion de 'listings' para el codigo de Python
\definecolor{codegreen}{rgb}{0,0.6,0}
\definecolor{codegray}{rgb}{0.5,0.5,0.5}
\definecolor{codepurple}{rgb}{0.58,0,0.82}
\definecolor{backcolour}{rgb}{0.95,0.95,0.92}

\lstdefinestyle{mystyle}{
    backgroundcolor=\color{backcolour},
    commentstyle=\color{codegreen},
    keywordstyle=\color{magenta},
    numberstyle=\tiny\color{codegray},
    stringstyle=\color{codepurple},
    basicstyle=\footnotesize\ttfamily,
    breakatwhitespace=false,
    breaklines=true,
    captionpos=b,
    keepspaces=true,
    numbers=left,
    numbersep=5pt,
    showspaces=false,
    showstringspaces=false,
    showtabs=false,
    tabsize=2
}
\lstset{style=mystyle}


%%%%%%%%%%%%%%%%%%%%%%%%%%%%%%%%%%%%%%%%%%%%%%%%%%%%%%%%%%%%%%%%%%%%%%%%%%%%%%%%%%%%
\begin{document}

% --- CONFIGURACION DE PIES DE FOTO ---
\captionsetup[figure]{margin=1.5cm,font=small,name={Figura},labelsep=colon}
\captionsetup[table]{margin=1.5cm,font=small,name={Tabla},labelsep=colon}

\frontmatter % Inicio de las paginas preliminares (portada, indice)

%%%%%%%%%%%%%%%%%%%%%%%%%%%%%%%%%%%%%%%%%%%%%%%%%%%%%%%%%%%%%%%%%%%%%%%%%%%%%%%%
% --- PORTADA ---
\begin{titlepage}
    \begin{center}
        % IMPORTANTE: Colocar el archivo 'UDLogo.png' en la carpeta 'latex_src/figures/'
        \includegraphics[width=6cm]{figures/UDLogo.png}\\[0.5cm]
        {\LARGE Universidad Distrital Francisco José de Caldas\\[0.3cm]
        Facultad de Ingeniería\\[0.3cm]
        Ingeniería de Sistemas}\\[2.5cm]

        \linespread{1.0}\huge \textbf{
            Prueba de Rangos con Signo de Wilcoxon para Observaciones Pareadas
        }
        \linespread{1}~\\[2.5cm]

        {\Large
            Álvarez Ortiz Arley Santiago - 20241020008 \\
            Martínez Pardo Silvana - 20241020010 \\
            Moreno Granado Sergio Leonardo - 20242020091 \\
            Rodríguez Camacho Juan Esteban - 20241020029
        }\\[2cm]

        {\Large
            \emph{Docente:} Diego Alberto Chitiva Huertas, M.Sc.}\\[1.5cm]

        \large Un reporte presentado para el Proyecto Final de Probabilidad y Estadística\\
        Semestre 2025-II % Ajustar si es necesario
        \\[0.3cm]
        \vfill

        Diciembre de 2025, Bogotá D.C.
    \end{center}
\end{titlepage}

\mainmatter % Inicio del contenido principal del documento

\tableofcontents
\newpage

% --- INTRODUCCION ---
\section{Introducción}
\label{sec:introduccion}
En este apartado se debe delimitar el tema... (Texto de ejemplo)
% (El resto del contenido de las secciones se mantiene igual)

% --- DESARROLLO ---
\section{Desarrollo}
\label{sec:desarrollo}
Esta es la sección principal...
\subsection{Fundamentación Teórica}
Aquí se debe explicar la teoría...
Citamos nuestra fuente de ejemplo \parencite{wackerly2008}.

% --- ALGORITMO ---
\section{Algoritmo}
\label{sec:algoritmo}
En esta sección se presenta la implementación...
\begin{lstlisting}[language=Python, caption=Algoritmo de Wilcoxon en Python]
# El codigo de ejemplo se mantiene...
import numpy as np
from scipy.stats import wilcoxon

def prueba_wilcoxon_pareada(muestra1, muestra2, alpha=0.05):
    pass # Implementacion
\end{lstlisting}

% --- CONCLUSION ---
\section{Conclusión}
\label{sec:conclusion}
En la conclusión, deben resumir...

% --- BIBLIOGRAFIA ---
\printbibliography[title={Bibliografía}]


\end{document}
